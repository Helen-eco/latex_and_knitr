% !TEX TS-program = xelatex
% !TEX encoding = UTF-8 Unicode


\documentclass{article}
\usepackage{xltxtra}
\usepackage{fontspec}
\usepackage{polyglossia}
% \usepackage{minted}
\setmainlanguage[babelshorthands=true]{russian}
\usepackage[usenames,dvipsnames,svgnames,table]{xxcolor}
%\usepackage{calc}


% babel shorthands are:
%  "--- Cyrillic emdash in plain text.
%  "--~ Cyrillic emdash in compound names (surnames).
%  "--* Cyrillic emdash for denoting direct speech.
% "~ unbreakable hype. Example: как в 90"~х годах

\setotherlanguage{english}



\setmainfont{Cambria}
%\setromanfont{Times New Roman}
\setsansfont{Arial}
\setmonofont{Courier New}



% \newfontfamily{\cyrillicfontrm}{Times New Roman}
\newfontfamily{\cyrillicfonttt}{Courier New} 


\begin{document}
	Привет, \LaTeX!
    
    \newcommand{\txtscript}[1]{\textit{\scriptsize{}#1}}
    
    \[
    \fbox{$\rho_\txtscript{молока}=\frac{M_\txtscript{молока}}{V_\txtscript{молока}}$}
    \]
   %  \mint{C}|for(i=0;i<10;++i) printf("%d",i) ;|
     
     \newcounter{nomer}
     \renewcommand{\thenomer}{\asbuk{nomer}}
     \setcounter{nomer}{5}
     \Asbuk{nomer}
     
     \asbuk{nomer}
     \addtocounter{nomer}{2*3}
     \arabic{nomer}
     \alph{nomer}
     
     \Roman{nomer}
     
     \thenomer
     \stepcounter{nomer}
     \thenomer \alph{nomer}
     
     
     \begin{enumerate}
      
     \item пунткт \arabic{enumi}    
     \item пунткт \arabic{enumi}
     \item пунткт \arabic{enumi}
     \end{enumerate}
     
     %     \vskip{0.1\textheight}
     
 \colorlet{barcolor1}{red!20!yellow!15}

 \colorlet{barcolor2}{blue!20!red!15}

\newlength{\barlength}
\setlength{\barlength}{4 cm minus 1 cm}
\definecolor{gray50}{gray}{0.5}
 \noindent
  \textcolor{barcolor1}{\rule{\barlength}{1cm}}
 \rule{\barlength}{1pt}
 \vspace{\barlength}
  \textcolor{barcolor2}{\rule{\barlength}{1cm}}
  \textcolor{gray50}{\rule{\barlength}{1 cm}}
    
%         \begin{theorem}{О усталости}
%         Каждый когда-нибудь устанет\ldots
%        \end{theorem}
		
        
        
   \begin{itemize}
       \item Пункт 
     \end{itemize}   
        
    \begin{description}
        \item[First] \hfill \\
        The first item
        \item[Second] \hfill \\
        The second item
        \item[Third] \hrulefill \\
        The third etc \ldots
    \end{description}    
        
        
\end{document}