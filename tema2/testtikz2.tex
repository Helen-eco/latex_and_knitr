\documentclass{article} 
\usepackage{fontspec}


\usepackage{amsmath,amssymb,amsthm}


\usepackage{polyglossia}   %% загружает пакет многоязыковой вёрстки
\setdefaultlanguage[spelling=modern,babelshorthands=true]{russian}  %% устанавливает главный язык документа
\setotherlanguage{english} %% объявляет второй язык документа
\defaultfontfeatures{Ligatures={TeX},Renderer=Basic}  %% свойства шрифтов по умолчанию
\setmainfont[Ligatures={TeX,Historic}]{Times New Roman} %% задаёт основной шрифт документа



\setsansfont{Arial}
\setmonofont{Courier New}


\newfontfamily{\cyrillicfontrm}{Times New Roman}
\newfontfamily{\cyrillicfontsf}{Arial}
\newfontfamily{\cyrillicfonttt}{Courier New}

\usepackage{verbatim}

\usepackage{xcolor}
\newtheorem{theorem}{Теорема}

\usepackage{tikz}
\usetikzlibrary{shapes,calc,positioning,graphs, arrows,patterns,through}


%\definecolor{gray50}{gray}{0.5}
\definecolor{myblue}{RGB}{12,131,200}
\definecolor{myred}{RGB}{200,100,100}
\definecolor{mydeepblue}{HTML}{1F2FAA}

\colorlet{mymix}{yellow!40!magenta}


\begin{document}
	
%	\textcolor{gray50}{текст}
	
	
	

    \begin{theorem}
        Всё хорошее когда-нибудь кончается.
        
     \end{theorem}   
    
    
        \begin{verse}
            Однажды в студёную зимнюю пору \\
            Сижу за решёткой в темнице сырой \\
            Гляжу подинмается медленно в гору \\
            Вскорменный  в неволе орел молодой.  \\[3ex]
            И шествуя важно походкою чинной \\ 
            Мой верный товарищ махая крылом \\
            В больших сапогах в полушубке овчинном \\
            Кровавую пищу клюёт под окном
            
        \end{verse}
   
    
%\tikzset{everypicture/.style=thick, rounded corners=5}
    
    
   \tikz %
   \draw [blue] (1, 0) -- ++(1 cm , 0) -- +(0 , 1cm ) -- ++(1 cm , 0);    


\vspace{0.5cm}

\begin{tikzpicture}[color=mymix,]
  \node(ЮгЗап) at (-1,-1) {Привет!} ;  
   \filldraw [draw=myblue,very thick] (ЮгЗап) rectangle((5 , 2);
   

   
  
   
\end{tikzpicture}    
  
%  $ \foreach \i / \k in {1/a, 2/b, 3/c} {( \i: \k) } $
   
  % \foreach \i / \j / \k in {1/a, 2/b, 3/c} {( \i: \k) }
   
    \foreach \i  in {1/a, 2/b, 3/c} {( \i: afa) }
    
    
    
\foreach \i in {1,2, ..., 4} {\i --} \\
 \foreach \i in {4 , ..., 1} {\i \ $ \rightarrow $ } \\
\foreach \i in {a, ..., d} {\i , } \\
\foreach \i in {1, 1.5 , ..., 5.8} {-\i -} 

% проверьте пакет amsmath
 $ \foreach \i / \j in {1/a, 2/b, 3/c}  {\dfrac{\i}{\j} }$    
    
    
 \foreach \i [ evaluate = \i as \result ] %
in {1 + 2, 10 * 3, 2^3} %
{$ \i = \result \quad $} %
    
\begin{figure}
\centering
\begin{tikzpicture}[scale=3]
\draw [thick, green!40!black ] %
(0, 0) -- +(0:1 cm) -- +(60:1 cm) -- cycle ;
\end {tikzpicture}




\caption {a picture in a figure }
\end {figure}



\tikz %
\draw %
(0, 0) circle [ radius = 5mm ];
\\[1 em]

\tikz %
\draw %
(0, 0) circle %
[ %
x radius = 5mm , %
y radius = 3mm %
];

\tikzstyle{flownode} = [font=\sffamily,shape=rectangle,rounded corners=10pt, outer sep=1mm, minimum size=2cm,dashdotted,left color=gray!10,right color=cyan!60] ;

\tikzstyle{flowtext} = [sloped,yshift=2mm,font={\small\itshape}] ;

\begin{figure}
\centering
\begin{tikzpicture}
\node (Банк) at (0,0) [flownode] {Банк};
\node (Завод) [flownode,above = 2cm of Банк, draw]{Завод} ;
\node (Потребитель) [flownode,left =  2cm of Банк , draw]{Потребитель} ;
\begin{scope} 

\draw [stealth-]  (Завод) edge 
%      node  [flowtext] {поток денег} 
       (Банк)  
      (Банк) edge 
            node  [flowtext]{деньги} 
                 (Потребитель)  ;
\draw [stealth-,red]  (Потребитель)  edge 
node  [flowtext] {материальный поток} 
(Завод) ;
                 
\end{scope}
\end{tikzpicture}

\end{figure}

\end{document}
