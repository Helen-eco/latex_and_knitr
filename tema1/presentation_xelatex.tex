% !TeX spellcheck = ru_RU
% !TeX encoding = UTF-8
% !TeX program = xelatex



\documentclass{beamer}

\usepackage{fontspec}
\usepackage{polyglossia}

\setmainfont{Times New Roman}
\setsansfont{Arial}
\setmonofont{Courier New}
\newfontfamily{\cyrillicfonttt}{Courier New}
\setdefaultlanguage{russian}


\usetheme{Luebeck}
  
\author{Илья Кочергин}
\title{Пробный документ}
\date{\today}
\institute{экономический факультет МГУ}

\begin{document}
  
\begin{frame}
 \titlepage	
\end{frame}  

\begin{frame}[fragile]
\frametitle{Формула}
 $$ M=\frac{V^{24}}{F_{ij}} $$
	Её код \LaTeX{}:
	\begin{verbatim}
	$$ M=\frac{V^2}{F_{ij} $$
	\end{verbatim}	
	
\end{frame}  

\begin{frame}
	\frametitle{списки}
Маркированный:	
	\begin{itemize}
		\item Продукты
		\item Хозтовары
		\begin{itemize}
			\item Для дома 
			\item Для дачи
		\end{itemize}
	\end{itemize}	
Нумерованный:
	\begin{enumerate}
		\item Продукты
		\item Хозтовары
		
		\begin{enumerate}
			\item Для дома 
			\item Для дачи
		\end{enumerate}
		
		
	\end{enumerate}
	
\end{frame}
\begin{frame}
	\frametitle{списки 2 столбца}
\begin{columns}[c]
\begin{column}{5cm}
	Маркированный:	
	\begin{itemize}
		\item Продукты
		\item Хозтовары
		\begin{itemize}
			\item Для дома 
			\item Для \alert{дачи}
			\end{itemize}
			\end{itemize}	
\end{column}

 \begin{column}{5cm}
	Нумерованный:
	\begin{enumerate}
		\item Продукты
		\item Хозтовары
		
		\begin{enumerate}
			\item Для дома 
			\item Для дачи
			\end{enumerate}
			
			
			\end{enumerate}
 \end{column}
\end{columns}
	
	
\end{frame}

\begin{frame}
	\frametitle{There Is No Largest Prime Number}
	\framesubtitle{The proof uses \textit{reductio ad absurdum}.}
	\begin{theorem}
		There is no largest prime number.
	\end{theorem}
	\begin{proof}
		\begin{enumerate}
			\item<1-| alert@{2,4}> Suppose $p$ were the largest prime number.
			\item<2-3> Let $\sqrt{q}$ be the product of the first $p$ numbers.
			\item<3-> Then $q+1$ is not divisible by any of them.
			\item<1-> But $q + 1$ is greater than $1$, thus divisible by some prime
			number not in the first $p$ numbers.\qedhere
		\end{enumerate}
	\end{proof}
\end{frame}
	
	
\end{document} 


  Привет!
  
  {
    \sffamily
    Рубленый текст	
  }
  
  {\ttfamily
  текст напечатан на машинке	
  }
  


  
  \[ M=\frac{V}{F} \]
  
 			
 
      

 
