% !TEX encoding = UTF-8
%!TEX program = pdflatex
% !TeX spellcheck = ru_RU
% Пример оформления документа в LuaLaTeX с использованием polyglossia

\documentclass[10pt,pdf]{beamer}



\usepackage[utf8]{inputenc}
\usepackage[T2A]{fontenc}
\usepackage[english,russian]{babel} % поддержка языков 
 \usepackage[normalem]{ulem} % выдление подчеркиванием и зачеркиванием (\uline{}, \uwave{})
% но стандартную команду \emph{text} не переопределяем


% babel shorthands are:
%  "--- Cyrillic emdash in plain text.
%  "--~ Cyrillic emdash in compound names (surnames).
%  "--* Cyrillic emdash for denoting direct speech.
% "~ неразрывный дефис. Пример: как в 90"~х годах


\usepackage{tikz}
\usetikzlibrary{graphs,shapes,arrows}


\usetheme{Warsaw}


\title{LaTeX+R=Документ Knitr}
\author{Илья Кочергин}
\institute{кафедра экономической информатики ЭФ МГУ} 
\usepackage{Sweave}
\begin{document}
\Sconcordance{concordance:knitr_beamer_pdflatex.tex:knitr_beamer_pdflatex.Rnw:%
1 56 1 50 0 1 3 6 1 11 0 4 1}

\begin{frame}

\titlepage
\end{frame}	
\section[простые примеры]{Примеры с простыми chunk'ами}
\begin{frame}
\frametitle{Простой слайд}
 Просто текст слайда 
 \begin{itemize}
 
 \end{itemize}
 

\end{frame}


\begin{frame}
	\frametitle{Необходимость согласовывать и корректировать}
\begin{itemize}
\item 	Договоры 4
\item 	Тексты учебных программ
\item 	Дипломные работы 
\item 	Создание коллективного  труда
\end{itemize} 
\end{frame}


\begin{frame}[fragile]
\frametitle{Второй слайд с кодом R}    

\begin{Schunk}
\begin{Sinput}
>   ggplot(mapp=aes(x=c(0,6)))+
+   stat_function(fun = sin)+
+   ggtitle("Привет!")
\end{Sinput}
\end{Schunk}

\end{frame}

\end{document}
