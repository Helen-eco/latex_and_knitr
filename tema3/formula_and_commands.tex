% !TEX TS-program = pdflatex
% !TEX encoding = UTF-8 Unicode


\documentclass{article}

% !TeX program = pdflatex

\usepackage[utf8]{inputenc}
\usepackage[T2A]{fontenc}
\usepackage[english,russian]{babel}
\usepackage{mathtools,amstext,amsfonts}
\usepackage{verbatim}
\usepackage{url}
\usepackage[usenames,dvipsnames,svgnames,table]{xcolor}
\usepackage{ntheorem}






\usepackage{amstext}
\usepackage{calc}
\usepackage{framed}
\usepackage{parskip}
\usepackage{mathtools}



\begin{document}


\section{Примеры определения команд}
        Привет, \LaTeX! \\
        \newcommand{\mytexlogo}{pdf\LaTeX} \\
        Привет, \mytexlogo! \\
        \renewcommand{\mytexlogo}{xe\LaTeX} \\
        Привет, \mytexlogo!
\begin{framed}
 
\begin{verbatim}
Привет, \LaTeX!
\newcommand{\mytexlogo}{pdf\LaTeX} \\
Привет, \mytexlogo! \\
\renewcommand{\mytexlogo}{xe\LaTeX} \\
Привет, \mytexlogo!   
\end{verbatim}

\end{framed}

\hrule

Вставляем произвольный текст 


\begin{verbatim}
\begin{equation}
    \rho_\text{молока}=\frac{M_\text{молока}}{V_\text{молока}} 
    \text{, если корова тяжелее воздуха}
\end{equation}
\end{verbatim}

\begin{equation}
    \rho_\text{молока}=\frac{M_\text{молока}}{V_\text{молока}} 
    \text{, если корова тяжелее воздуха} \\
\end{equation}

\hrule

    \newcommand{\txtscript}[1]{\textit{\scriptsize{}#1}}
    
    \begin{verbatim}
    \newcommand{\txtscript}[1]{\textit{\scriptsize{}#1}}
    \end{verbatim}

    \[
    \boxed{\rho_\txtscript{молока}=\frac{M_\txtscript{молока}}{V_\txtscript{молока}}} 
    \]
     
\begin{verbatim}
     \[
    \boxed{\rho_\txtscript{молока}=\frac{M_\txtscript{молока}}{V_\txtscript{молока}}} 
    \]
\end{verbatim}

    \renewcommand{\txtscript}[1]{\textsf{\scriptsize{}#1}}
    
    \begin{verbatim}
    \renewcommand{\txtscript}[1]{\textsf{\scriptsize{}#1}}
    \end{verbatim}

\newcommand{\myind}{\txtscript{молока}}


    \[
    \boxed{\rho_\myind=\frac{M_\myind}{V_\myind}}
    \]
 
\section{Счётчики}


\verb|\newcounter{nomer} | 
\newcounter{nomer}

\verb|\thenomer | 
\thenomer

\stepcounter{nomer}
\verb|     \stepcounter{nomer} |

 
\verb|\thenomer | 
\thenomer


\verb|     \renewcommand{\thenomer}{\alph{nomer}}   |
     \renewcommand{\thenomer}{\alph{nomer}}   

 
\verb|\thenomer | 
\thenomer

\stepcounter{nomer}
\verb|     \stepcounter{nomer} |


\verb|\thenomer | 
\thenomer
 
\verb|\thenomer | 
\thenomer


\verb|     \setcounter{nomer}{5} |
     \setcounter{nomer}{5}

\verb|     \Asbuk{nomer}  |
     \Asbuk{nomer}
     
\verb|     \asbuk{nomer} |
     \asbuk{nomer}
     
     \addtocounter{nomer}{3}
\verb|     \addtocounter{nomer}{3} |

\verb|         \arabic{nomer}   |
     \arabic{nomer}

\verb|         \alph{nomer}  |
     \alph{nomer}
     
\verb|         \Roman{nomer}   |
     \Roman{nomer}
     
\verb|           \thenomer  |
     \thenomer

\verb|           \stepcounter{nomer}  |
     \stepcounter{nomer}

\verb|           \thenomer - \alph{nomer} |
     \thenomer - \alph{nomer}
     
\section{Списки}

     \begin{enumerate}
      
     \item пунткт \arabic{enumi}    
     \item пунткт \arabic{enumi}
     \item пунткт \arabic{enumi}
     \end{enumerate}
     
Список с римскими номерами

\begin{verbatim}
   \begin{enumerate}
     \renewcommand{\theenumi}{\Roman{enumi}} 
     \item пунткт \arabic{enumi}    
     \begin{enumerate}
      \item подпункт \arabic{enumii}
      \item подпункт \arabic{enumii}
      \item подпункт \arabic{enumii}
     \end{enumerate}
 
     \item пунткт \arabic{enumi}
     \begin{enumerate}
     \renewcommand{\theenumii}{\asbuk{enumii}} 
      \item подпункт \arabic{enumii}
      \item подпункт \arabic{enumii}
      \item подпункт \arabic{enumii}
     \end{enumerate}
 
     \item пунткт \arabic{enumi}
  \end{enumerate}
\end{verbatim}

   \begin{enumerate}
     \renewcommand{\theenumi}{\Roman{enumi}} 
     \item пунткт \arabic{enumi}    
     \begin{enumerate}
      \item подпункт \arabic{enumii}
      \item подпункт \arabic{enumii}
      \item подпункт \arabic{enumii}
     \end{enumerate}
 
     \item пунткт \arabic{enumi}
     \begin{enumerate}
     \renewcommand{\theenumii}{\asbuk{enumii}} 
      \item подпункт \arabic{enumii}
      \item подпункт \arabic{enumii}
      \item подпункт \arabic{enumii}
     \end{enumerate}
 
     \item пунткт \arabic{enumi}
  \end{enumerate}
     

 \section{newtheorem}    
     
     
 \colorlet{barcolor1}{red!20!yellow!15}

 \colorlet{barcolor2}{blue!20!red!15}

\newlength{\barlength}
\setlength{\barlength}{4 cm minus 1 cm}
\definecolor{gray50}{gray}{0.5}
 \noindent
  \textcolor{barcolor1}{\rule{\barlength}{1cm}}
 \rule{\barlength}{1pt}
 \vspace{\barlength}
  \textcolor{barcolor2}{\rule{\barlength}{1cm}}
  \textcolor{gray50}{\rule{\barlength}{1 cm}}
    
%         \begin{theorem}{О усталости}
%         Каждый когда-нибудь устанет\ldots
%        \end{theorem}
		
        
        
   \begin{itemize}
       \item Пункт 
     \end{itemize}   
        
    \begin{description}
        \item[First] \hfill \\
        The first item
        \item[Second] \hfill \\
        The second item
        \item[Third] \hrulefill \\
        The third etc \ldots
    \end{description}    
        
        
\end{document}